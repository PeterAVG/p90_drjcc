\documentclass[conference]{IEEEtran}
\IEEEoverridecommandlockouts
% The preceding line is only needed to identify funding in the first footnote. If that is unneeded, please comment it out.

\pagenumbering{gobble}


\usepackage{cite}
\usepackage{amsmath,amssymb,amsfonts}
%\usepackage{algorithmic}
\usepackage{algorithm}
% \usepackage{acro}
\usepackage[printonlyused,nolist]{acronym}
\usepackage{algpseudocode}
\usepackage{graphicx}
\usepackage{textcomp}
\usepackage{xcolor}
\usepackage{adjustbox}
\usepackage[version=4]{mhchem}
\usepackage{booktabs}
\usepackage{multirow}
\usepackage{siunitx}
\usepackage{eurosym}
\usepackage{mhchem}
\usepackage{amssymb}
\usepackage{tikz}
\usepackage{pgfplots}
\pgfplotsset{compat=1.7}
\usepackage{pgfplotstable}
\usepgfplotslibrary{fillbetween}
\pgfdeclarelayer{bg}    % Background layer
\pgfdeclarelayer{main}  % Main layer
\usepgfplotslibrary{groupplots} % Load the groupplots library
% \pgfplotsset{compat=newest}

\usepackage{multirow}
\usepackage{graphicx}
\usepackage{hyperref}
\usepackage{eurosym}
\usepackage{subcaption}
\def\BibTeX{{\rm B\kern-.05em{\sc i\kern-.025em b}\kern-.08em
    T\kern-.1667em\lower.7ex\hbox{E}\kern-.125emX}}
\usepackage{url}

\newtheorem{theorem}{Theorem} % Required for theorem environments
\newtheorem{definition}{Definition} % Define a new theorem environment for definitions


\newcommand{\red}[1]{\textcolor{red}{#1}}
\newcommand{\figspace}{\vspace{-0.5cm}}
\newcommand{\paragraphspace}{\vspace{-0.3cm}}



\begin{document}




\title{Distributionally Robust Optimization in Action for Bidding in Nordic Ancillary Service Markets}

\author{Peter A.V. Gade\textsuperscript{*}\textsuperscript{\textdagger}, Henrik Bindner\textsuperscript{*}, Jalal Kazempour\textsuperscript{*} \\
    \textsuperscript{*}Department of Wind and Energy Systems, Technical University of Denmark, Kgs. Lyngby, Denmark \\
    \textsuperscript{\textdagger}IBM Client Innovation Center, Copenhagen, Denmark
    % <-this % stops a space
    \thanks{
        %Corresponding author. Tel.: +45 24263865. \\
        Email addresses: pega@dtu.dk (P.A.V. Gade), hwbi@dtu.dk, jalal@dtu.dk (J. Kazempour).}% <-this % stops a space
    \vspace{-3mm}
}

% \footnote{Corresponding author. Tel.: +45 24263865. \\ Email addresses: pega@dtu.dk (P.A.V. Gade), jalal@dtu.dk (J. Kazempour).}

% The paper headers
%\markboth{Journal of \LaTeX\ Class Files,~Vol.~14, No.~8, August~2021}%{Shell \MakeLowercase{\textit{et al.}}: A Sample Article Using IEEEtran.cls for IEEE Journals}

%\IEEEpubid{0000--0000/00\$00.00~\copyright~2021 IEEE}
% Remember, if you use this you must call \IEEEpubidadjcol in the second
% column for its text to clear the IEEEpubid mark.

\maketitle

% \tableofcontents

\IEEEaftertitletext{\vspace{-0.8\baselineskip}}
\maketitle
\thispagestyle{plain}
\pagestyle{plain}
\begin{abstract}
    Stochastic flexible resources can now offer their capacity to Nordic ancillary services under a new set of regulations that addresses their stochastic nature. In this work, we show how aggregators can exploit the market regulations while by generalizing the rule in mathematical representation that is distributionally robust with respect to their uncertain offering capacity. A tractable formulation is provided in the form of a \ac{MILP}, and it is agnostic to the underlying technology offering its flexibility. The Nordic \acp{TSO} can subsequently investigate how aggregators react to their stated allowed violation frequency within their regulations. We show how a \ac{TSO} can maximize its available flexible capacity using a bi-level optimization problem where the inner problem represents the aggregators' profit maximization. This is illustrated in a simulated case study of an aggregator with a portfolio of \acp{EV}.
\end{abstract}

\begin{IEEEkeywords}
    % Synergy effect, demand-side flexibility, EV-chargers, joint chance constraints, ancillary services
    Distributionally robust optimization, Joint chance-constraints, ancillary services
\end{IEEEkeywords}


% ############## OBS: MAX 10 PAGES!!!! ##############

\vspace{-2mm}

% TODO: delete
\section{Acronyms}\label{sec:acronyms}


\begin{acronym}[DRJCCP]
    \acro{IoT}{Internet of Things}
    \acro{OPF}{Optimal Power Flow}
    \acro{IEEE}{Institute of Electrical and Electronic Engineers}
    \acro{CHP}{Combined Heat \& Power}
    \acroplural{CHP}[CHPs]{Combined Heat \& Power}
    \acro{SOCP}{Second-Order Cone Program}
    \acroplural{SOCP}[SOCPs]{Second-Order Cone Programs}
    \acro{MILP}{Mixed-Integer Linear Program}
    \acro{MINLP}{Mixed-Integer Nonlinear Program}
    \acro{LP}{Linear Program}
    \acro{AC}{alternating current}
    \acroplural{LP}[LPs]{Linear Programs}
    \acro{COP}{Coefficient of Performance}
    \acro{KKT}{Karush-Kuhn-Tucker}
    \acro{DSO}{Distribution System Operator}
    \acroplural{DSO}[DSOs]{Distribution System Operator}
    \acro{TSO}{Transmission System Operator}
    \acroplural{TSO}[TSOs]{Transmission System Operator}
    \acro{DER}{Distributed Energy Resource}
    \acroplural{DER}[DERs]{Distributed Energy Resources}
    \acro{TCL}{Thermostatically Controlled Load}
    \acroplural{TCL}[TCLs]{Thermostatically Controlled Loads}
    \acro{BRP}{Balance Responsible Party}
    \acroplural{BRP}[BRPs]{Balance Responsible Parties}
    \acro{NEMO}{Nominated Electricity Market Operator}
    \acroplural{NEMO}[NEMOs]{Nominated Electricity Market Operators}
    \acro{FCR}{Frequency Containment Reserve}
    \acro{FCR-N}{Frequency Containment Reserve - Normal Operation}
    \acro{FCR-D}{Frequency Containment Reserve - Disturbance}
    \acro{FFR}{Fast Frequency Reserve}
    \acro{mFRR}{Manual Frequency Restoration Reserve}
    \acro{aFRR}{Automatic Frequency Restoration Reserve}
    \acro{ENTSO-E}{European Network of Transmission System Operators for Electricity}
    \acro{TERRE}{Trans-European Replacement Reserves Exchange}
    \acro{MARI}{Manually Activated Reserves Initiative}
    \acro{PICASSO}{Platform for the International Coordination of Automated Frequency Restoration and Stable System Operation}
    \acro{ODE}{Ordinary Differential Equation}
    \acroplural{ODE}[ODEs]{Ordinary Differential Equations}
    \acro{PDE}{Partial Differential Equation}
    \acroplural{PDE}[PDEs]{Partial Differential Equations}
    \acro{SDE}{Stochastic Differential Equation}
    \acroplural{SDE}[SDEs]{Stochastic Differential Equations}
    \acro{NODE}{Neural Ordinary Differential Equation}
    \acroplural{NODE}[NODEs]{Neural Ordinary Differential Equations}
    \acro{PINN}{Physics-Informed Neural Network}
    \acroplural{PINN}[PINNs]{Physics-Informed Neural Networks}
    \acro{PID}{Propportional Integral Derivative}
    \acro{MPC}{Model Predictive Control}
    \acro{E-MPC}{Economic Model Predictive Control}
    \acro{OD}{Opening Degree}
    \acro{IS}{in-sample}
    \acro{OOS}{out-of-sample}
    \acro{CC}{Chance Constraint}
    \acroplural{CC}[CCs]{Chance Constraints}
    \acro{JCC}{Joint Chance Constraint}
    \acroplural{JCC}[JCCs]{Joint Chance Constraints}
    \acro{DRJCC}{Distributionally Robust Joint Chance Constraint}
    \acroplural{DRJCC}[DRJCCs]{Distributionally Robust Joint Chance Constraints}
    \acro{DRJCCP}{Distributionally Robust Joint Chance Constraint Program}
    \acro{EV}{Electric Vehicle}
    \acroplural{EV}[EVs]{Electric Vehicles}
    \acro{CVaR}{Conditional Value at Risk}
    \acro{NP}{Non-deterministic Polynomial-time}
    \acro{SAA}{Sample Average Approximation}
    \acro{EV}{Electric Vehicle}
    \acroplural{EV}[EVs]{Electric Vehicles}
    \acro{LER}{Limited Energy Reservoir}
    \acroplural{LER}[LERs]{Limited Energy Reservoirs}
    \acro{PV}{Photovoltaic}
    \acroplural{PV}[PVs]{Photovoltaics}
\end{acronym}


\section{Introduction}\label{sec:Introduction}



\subsection{Background}\label{sec:background}

% \ac{DRJCC} for CCH EVs Nordic ancillary service markets. Focus is not on P90 or EV/technology

Flexible resources with stochastic production or demand have the capability to balance the power grid, but their stochastic nature has also proved challenging for their integration into ancillary services. In this work, we show how Nordic ancillary service market rules naturally allow for bidding flexible resources in a way that can be represented using \acp{DRJCC}. The methodology is agnostic to technology and markets. We show the tradeoff between increased supply from stochastic flexible resources and uncertainty of delivery. For the \acp{TSO} and a system perspective, this is of particular interest, while from an aggregator\footnote{For the remainder of this paper, we refer to an aggregator as any entity that offers flexible capacity from at least one technology or asset into some ancillary service. Aggregator is thus used interchangeably with flexible provider.} perspective, the methodology generalizes bidding as being either distributionally robust or empirically robust.

Nordic ancillary markets are under being developed to incentive stochastic flexible resources to offer their capacity. This is mainly due to two reasons: \textit{(i)} the Nordic power grid is undergoing a rapid transition with less fossil-fuel based generation and more intermittent generation from wind and solar power, thus stressing the power grid. And \textit{(ii)} the demand for flexible capacity from the Nordic \acp{TSO} has increased with a particular focus on flexibility procurement from \textit{green} technologies, e.g., batteries, wind and solar power, and flexible demand. Therefore, the Danish \ac{TSO} has prescribed an innovative regulation, called the \textit{P90} rule, which specifically addresses how such stochastic flexible resources should offer their capacity into Nordic ancillary services.

The P90 rule addresses the stochastic nature of such flexible resources with respect to their unknown future flexible capacity. Currently, bidding for Nordic ancillary services generally occurs in the day-head stage \cite{energinet}, and stochastic resources can not know with certainty their baseline power consumption 12-36 hours in advance. Hence, the Danish \ac{TSO} allows for an allowed violation of bids of 10\% (as defined later in Section \ref{sec:problem-formulation}).

This rule naturally emits a mathematical representation of an aggregator offering flexible capacity using \acp{CC}. Thus, the aggregator can exploit this rule to its advantage by maximizing profits within the allowed violation range. However, this might not be in the aggregators, or \ac{TSO}'s, best interest as we show in this paper. Any mis-specification of the aggregators empirical distribution of offering capacity might cause a violation of the Nordic market rules and result in too little available capacity for the \ac{TSO} in real-time. We therefore investigate how a distributionally robust representation of this rule can benefit aggregators when bidding capacity under such non-stationary environments, while the \ac{TSO} can use this information to assess different allowed violation frequencies to maximize its capacity procurement.

\subsection{Research questions and our contributions}

\acp{TSO} are looking to integrate more fossil-fuel free power flexibility, and flexible demand or production have the potential to provide some the supply to meet the \ac{TSO} demand. In order to integrate such stochastic flexible resources, aggregators \acp{TSO} face a number of challenges, respectively: \textit{(i)} Which level of conservativeness to choose when bidding flexible stochastic resources and \textit{(ii)} which violation frequency and conservativeness to choose in order to balance more supply from stochastic flexible resources as opposed to increased uncertainty of delivery and \textit{(iii)} increased supply, and therefore liquidity, decreases \ac{TSO} procurement costs.

In the first research question, we formulate an aggregator bidding problem as a \ac{DRJCCP} to represent Nordic market rules and to maximize profits. As an example of a flexible resource, we use a portfolio of \acp{EV}, simulating their power consumption patterns when charging. We note that the presented methodology applies for any stochastic flexible resource. The monetary value of bidding at different conservativeness levels is demonstrated by adjusting the Wasserstein radius in the \ac{DRJCC}. In particular, we show how the conservativeness level directly impacts the aggregators security of supply in non-stationary environments, e.g., a change of \ac{EV} consumption pattern.

In the second research question, it is shown how the tradeoff between supply and uncertainty of delivery is associated using a bi-level optimization problem where the outer problem represents a \ac{TSO} setting the violation frequency, i.e., the so-called \textit{P90} level, and conservativeness level. The inner problem represents an aggregator maximizing profit using the same \ac{DRJCCP} from (i). Thus, we assume the \ac{TSO} can not only change the allowed violation frequency, but also demand that flexible providers use a given level of conservativeness for robustness.

We do not consider any specific market (or prices) and thus leave the third research question for future work. We do assume, however, that flexible providers are operationally and technically capable of responding, e.g., that a portfolio of \acp{EV} can deliver a frequency response. Furthermore, activation of any flexibility is assumed to contain negligible energy delivery, thus simplifying the objective of aggregators and \acp{TSO} by ignoring balancing energy. Hence, our work is primarily applicable to Nordic frequency markets, but agnostic to any stochastic flexible resource.

\subsection{Status quo}

A plethora of studies in the literature have investigated how stochastic flexible resources can participate in various ancillary service markets, both with respect to flexible demand  \cite{bondy2016procedure, bondy2014performance, biegel2014integration, AchievingControllabilityofElectricLoads} and flexible production from intermittent resources \cite{hansen2016provision, ullah2009wind, morey2023comprehensive, alshehri2019modelling}. However, no studies have looked specifically at the Nordic market regulations, i.e., the P90 rule, when bidding stochastic flexible resources. Reference \cite{zhang2018data} investigate how distributed energy resources can offer their flexible capacity using \acp{CC}, but does not consider the P90 rule when formulating their optimization problem. Other studies have extensively used \acp{CC} or \acp{JCC} in the context of power systems. In \cite{guo2020chance}, the authors show how the energy and reserve market can be cleared using \acp{CC} in a P2P setting. The same approach was taken in \cite{bienstock2014chance} for optimal power flow. In \cite{roald2016optimization}, a comprehensive study evaluates how risk in power systems can be modelled with \acp{CC} and \acp{JCC}. Furthermore, our work does assume any specific technology for offering flexible capacity. Instead, we provide a general problem formulation in form of a \ac{DRJCCP} for aggregators or flexible providers to offer capacity of their stochastic flexible resources while adhering to specific Nordic market regulations.

In this work, we also show how the \ac{TSO} can procure flexibility using bi-level optimization problem with the inner problem being that of flexible demands offering capacity. We believe this is the first work to state such such a problem. In reference \cite{sheikhahmadi2021bi}, the authors use a bi-level problem to coordinate flexibility procurement between the \ac{DSO} and \ac{TSO}, but they do not look at how stochastic flexible resources react to changing market regulations. There is several studies on \ac{TSO}-\ac{DSO} coordination \cite{givisiez2020review, jiang2022flexibility}, but no studies on \ac{TSO}-aggregator dynamics.

\subsection{Paper organization}

The rest of the paper is organized as follows. First, we describe our problem formulation of bidding into Nordic ancillary service markets for stochastic flexible loads. We also formally define the \textit{P90} rule from the regulations and show how it naturally corresponds to a \ac{JCC} from which flexible providers can use to model and subsequently bid their flexibility. We also introduce the \ac{TSO} perspective of maximizing flexible capacity as a bi-level optimization problem. Second, we show results for a simulated case study of a portfolio of \acp{EV} that have the ability to adjust their power consumption. We describe the simulation setup and investigate how an aggregator can bid \ac{EV} portfolio flexibility for different conservativeness levels, i.e., Wasserstein distances in a \ac{DRJCCP}. We also show how the \ac{TSO}' procurement of flexibility changes for different violation frequencies and Wasserstein distances. Finally, we conclude the paper and discuss future work.

\section{Problem Formulation}\label{sec:problem-formulation}

In this section, we first describe how stochastic flexible resources can participate in Nordic ancillary service markets with respect to the \textit{P90} rule. We show how this corresponds to a simple optimization problem with one \ac{JCC}. We then proceed to make the distributionally robust with respect to the flexible resource's uncertain available flexibility and provide a suitable, tractable formulation in form of a \ac{MILP}. Lastly, we embed the bidding problem with the \ac{DRJCC} within a bi-level optimization where the outer problem is the \ac{TSO} perspective, and the inner problem is the aggregator perspective.

\subsection{P90 rule}

The Danish \ac{TSO}, Energinet, has released an innovative regulation that incentivizes participation of stochastic flexible loads in their ancillary service markets \cite{energinet}. It specifically introduces an allowed violation frequency of 10\% as stated below:

\begin{definition}[Energinet's P90 rule \cite{energinet}]\label{def:P90}
    This means, that the participant's prognosis, which must be approved by Energinet, evaluates that the probability is 10\% that the sold capacity is not available. This entails that there is a 90\% chance that the sold capacity or more is available. This is when the prognosis is assumed to be correct.

    The probability is then also 10\%, that the entire sold capacity is not available. If this were to happen, it does not entail that the sold capacity is not available at all, however just that a part of the total capacity is not available. The available part will with high probability be close to the sold capacity.
\end{definition}

Generally, bidding of ancillary services occur the day before delivery. The P90 rule in Definition \ref{def:P90} states that the bid (for a given hour) should be feasible at least 90\% of the time.\footnote{The period in question is usually an average of at least three months.} Thus, the rule allows for aggregators to (partially) fail in their forecast of their available flexibility. The second part of Definition \ref{def:P90} states that the magnitude of violations is not allowed to be extreme, thus discouraging severe overbids. Lastly, we note that there are additional requirements for flexible \ac{LER} units \cite{energinet} with which we do not consider here (but they can readily be included in all subsequent formulations).

We show in the next section, how the interpretation of rule naturally leads to an optimization problem of an aggregator maximizing profit with the rule defined as \ac{JCC}


\subsection{General formulation}

To embed Definition \ref{def:P90} in an optimization problem, we consider an aggregator maximizing profit by bidding flexibility into some Nordic ancillary service:

\begin{align}\label{P90:General}
    \max_{p_{h}^{\text{cap}}} \quad & \sum_h \lambda_h p_{h}^{\text{cap}}                                                                                                                                               \\
    \text{s.t.} \quad               & \mathbb{P}  \left( p_{h}^{\text{cap}} \leq P_{m}^{\text{B}}(\xi), \quad \forall{m} \in \mathcal{M}_{h},  \forall{h} \in \mathcal{H}  \right) \geq 1 - \epsilon \label{P90:General:jcc}
\end{align}

Here, we consider a given day with $\mathcal{H} = \{1, 2,  \ldots 24\}$ hours, $\mathcal{M} = \{1, 2,  \ldots 1440\}$ minutes. Further, $ \mathcal{M}_{h} = \{h \times 60 + m \mid m \in \{0, 1, 2, \ldots, 59\}\}$ represents all minutes in hour $h$, and $p_{h}^{\text{cap}}$ as the capacity bid for hour $h$. Note that the capacity bid could either represent symmetrical or unidirectional flexibility. The uncertainty lies in the baseline power consumption or production, $p_{m}^{\text{B}}(\xi)$, governed by $\mathcal{P}$. Thus, at the time of bidding, the aggregator is uncertain of its flexible resource's future baseline power. For example, for a portfolio of \acp{EV}, their future consumption is uncertain, but most likely predictable to some degree. The objective of \eqref{P90:General} is the profit of the aggregator and proportional to capacity bids and prices, $\lambda_h$.

The sole constraint, denoted as \eqref{P90:General:jcc} in \eqref{P90:General}, \ac{JCC}, which can be interpreted as follows: for each minute within the hour of a bid, there exists a probability of at least $1-\epsilon$ for successfully fulfilling the bid for the subsequent day. Here, $\epsilon = 0.10$, consistent with Definition \ref{def:P90}. Constraint \eqref{P90:General:jcc} is classified as a \ac{JCC} as it must be satisfied concurrently for all minutes throughout a day.

The variability in available flexibility for the subsequent day, denoted as $P_{m}^{\text{B}}(\xi)$, is subject to a probability distribution $\mathcal{P}$. Solving \eqref{P90:General} necessitates generating \textit{samples} of $P_{m}^{\text{B}}(\xi)$ by utilizing either historical data or a predictive distribution. Regardless of the method chosen, the resultant distribution may not accurately represent the true underlying uncertainty of the future power baseline. Consequently, the objective in this work and the following sections is \textit{not} to generate the most precise sample set - or the best prediction - of $P_{m}^{\text{B}}(\xi)$, but rather to identify optimal decisions based on a \textit{given} sample set, acknowledging any potential mis-specification of the sample set.

In Problem  \eqref{P90:General}, instead of assuming a single, empirically distribution, we can generalize to an ambiguity set of distributions within some distance of the empirical distribution. To do this, we introduce the Wasserstein distance between two distributions \cite{chen2022data},

\begin{align}\label{was}
    d_{\mathrm{W}}\left(\mathbb{P}_1, \mathbb{P}_2\right)=\inf _{\mathbb{P} \in \mathcal{P}\left(\mathbb{P}_1, \mathbb{P}_2\right)} \mathbb{E}_{\mathbb{P}}\left[\left\|\tilde{\boldsymbol{\xi}}_1-\tilde{\boldsymbol{\xi}}_2\right\|\right]
\end{align}

where $\|\cdot\|$ denote a norm, and $d_{\mathrm{W}}\left(\mathbb{P}_1, \mathbb{P}_2\right)$ represents the shortest distance probability mass transport between two marginal distributions, i.e., of $\mathbb{P}_1$ to $\mathbb{P}_2$ \cite{mohajerin2018data}. We can then define the ambiguity set with distributions of interest as defined by a preset Wasserstein distance, $\theta$:

\begin{align}\label{amb-set}
    \mathcal{F}(\theta)=\left\{\mathbb{P} \in \mathcal{P} \mid d_{\mathrm{W}}(\mathbb{P}, \hat{\mathbb{P}}) \leq \theta\right\}
\end{align}

Hence, when $\theta \approx 0$, the ambiguity set in \eqref{amb-set} looks at distributions very close to the empirical distribution (thus representing a regular \ac{JCC}). When $\theta >> 0$, the ambiguity set contains distributions very dissimilar to the empirical distribution, thus being more conservative. For example, a large $\theta$ considers many different baseline consumptions of a portfolio of \acp{EV}.

Finally, introducing the Wasserstein distance and parameter $\theta$, Constraint \eqref{P90:General:jcc} becomes distributionally robust, i.e., a \ac{DRJCC}:

\begin{subequations}\label{P90:General:DRJCC}
    \begin{align}
        \max_{p_{h}^{\text{cap}}} \quad & \sum_h \lambda_h p_{h}^{\text{cap}}                                                                                                                                                                                                     \\
        \text{s.t.} \quad               & \mathbb{P}  \left( p_{h}^{\text{cap}} \leq P_{m}^{\text{B}}(\xi), \quad \forall{m} \in \mathcal{M}_{h},  \forall{h} \in \mathcal{H}  \right) \notag  \\
        & \geq 1 - \epsilon \quad \forall{m} \in \mathcal{M}_{h},  \forall{h} \in \mathcal{H}, \forall{\mathbb{P}} \in \mathcal{F}(\theta) \label{P90:General:drjcc_c}
    \end{align}
\end{subequations}

Problem \eqref{P90:General:DRJCC} thus represents an aggregator bidding into a Nordic ancillary service under Definition \ref{def:P90} using a distributionally robust representation of its flexible resource's baseline power. It can be used by any aggregator or flexible provider bidding into Nordic ancillary service markets with stochastic flexibility, e.g., wind mills, \acp{PV}, electrolyzers, \acp{TCL}, \acp{EV}, etc.

In the following subsections, we introduce a tractable formulation, and thereafter show how a \ac{TSO} can optimize supply of flexible capacity from such stochastic resources.


\subsection{Tractable formulation}

Problem \eqref{P90:General} is generally intractable although analytical reformulations exist \cite{nemirovski2007convex}. However, Problem \eqref{P90:General:DRJCC} emits a tractable reformulation as given by \cite[Proposition 2]{chen2022data}:

\begin{subequations}\label{P90:General:DRJCC-tract}
    \begin{align}
        \max_{p_{h}^{\text{cap}}, q_i, s_i \geq 0, t} \quad & \sum_h \lambda_h p_{h}^{\text{cap}}                                                                                                                                    \\
        \text{s.t.} \quad                                   & \epsilon |\mathcal{I}| t - \sum_{i \in \mathcal{I}} s_i \geq \theta |\mathcal{I}|                                                                                      \\
                                                            & P_{m}^{\text{B}}(\xi_i) - p_{h}^{\text{cap}} + M\cdot q_i \geq t - s_i, \notag \\
                                                            &  \forall{m} \in \mathcal{M}_{h},  \forall{h} \in \mathcal{H},  \forall{i} \in \mathcal{I} \\
                                                            & M (1-q_i) \geq t - s_i                                                                                                                                                 \\
                                                            & q_i \in \{0,1 \}, \quad \forall{i} \in \mathcal{I}
    \end{align}
\end{subequations}

Problem \eqref{P90:General:DRJCC-tract} is a \ac{MILP} with binary variables, $q_{i}$, that indicate a violation for sample $i$. As mentioned, $\theta$ specifies how far samples from the empirical distribution can be transported, as shown in \eqref{amb-set}. The Big-M, $M$, is a finite positive constant and the reformulation is quite sensitive to its magnitude which becomes evident when $\theta = 0$: Here, $M$ constrains the maximum violation of the bid, causing the empirical distribution to lose its significance \cite[Theorem 2, Remark 1]{chen2022data}.

Moreover, it is not immediately obvious what value $\theta$ should take. Remember, the aggregator should set $\theta$ \textit{before} solving \eqref{P90:General:DRJCC-tract}. A high $\theta$ yields more conservative bids, lower profits, but more security of delivery. A low $\theta$ resembles the empirical distribution and finds the highest possible bids and profits, but this might prove sensitive to non-stationarity in the baseline power, $p_{m}^{\text{B}}(\xi)$. An option is to solve \eqref{P90:General:DRJCC-tract} for a predefined set of $\theta$ values and choose the smallest, feasible value of $\theta$, as suggested by \cite[Section 3.2]{chen2022data}.

Alternatively, the \ac{TSO} could also prescribe a value of $\theta$ according to its needs as shown in the next subsection.


\subsection{\ac{TSO} optimization problem}

The \ac{TSO} is primarily interested in procuring sufficient supply of capacity for its ancillary service markets. This is becoming increasing difficult with more intermittent, renewable generation in the power grid, but also with a pressure to not procure (too much) fossil-fuel based reserve capacity. Exactly for these reasons, the \ac{TSO} in Denmark has introduced the P90 rule as defined earlier in \ref{def:P90} to allow stochastic flexible resources to bid into ancillary service markets.

However, the violation frequency of $\epsilon = 0.10$ in Definition \ref{def:P90} is rather arbitrary and one could imagine other values of $\epsilon$ that yields more flexible capacity, although with less certainty of delivery. In this section, we show how to formulate a \ac{TSO} optimization problem over $\epsilon$ in response to how aggregators of flexible demands, $d \in \{\mathcal{D} \}$, behave according to \eqref{P90:General:DRJCC-tract}.

Furthermore, we assume the \ac{TSO} also prescribe a degree of conservativeness upon the aggregators by pre-setting $\theta$, and that it ignores the cost of procurement. Then \ac{TSO} capacity procurement is described by the following bi-level optimization problem:

\begin{subequations}\label{P90:TSO}
    \begin{align}
        \max_{\epsilon, \theta} \quad & \sum_h p_{h,d}^{\text{cap}} -  \frac{1}{60|\mathcal{I}|}\sum_{m,h,i,d} \nu_{m,h,i,d}                                                                                                                                                                                                      \\
        \text{s.t.} \quad               & \text{Problem} \thinspace \eqref{P90:General:DRJCC-tract}, \quad \forall{d} \label{P90:TSO:inner} \\
        & \nu_{m,h,i,d} = \left(p_{h,d}^{\text{cap}} - P_{m,d}^{\text{B}}(\xi_i) \right)^{+}, \quad \forall{i}, \forall{m}, \forall{h}, \forall{d}
    \end{align}
\end{subequations}

In the inner optimization problem of \eqref{P90:TSO}, as represented by Constraint \eqref{P90:TSO:inner}, each demand maximizes its profits as per \eqref{P90:General:DRJCC-tract} and decision variables $p_{h,d}^{\text{cap}}$. In the outer optimization problem, the \ac{TSO} optimizes over $\epsilon$ and $\theta$ which are seen as parameters to the flexible demands. The objective of the \ac{TSO} is to incentive demands to bid as much \textit{reliable} flexible capacity as possible, given that demands maximize profits by imposing distributional robustness on their uncertain baseline power using a \ac{DRJCC}. The variable, $\nu$ represents ex-post unavailable capacity from demands and is subtracted from their total supply of capacity.


\section{Simulation Results}

In this section, we show through a simulated case study two main findings: \textit{(i)} How an aggregator of flexible \acp{EV} can bid using \eqref{P90:General:DRJCC-tract} for three different conservativeness levels, $\theta$, with respect to \ac{IS} and \ac{OOS} performance, and \textit{(ii)} how a \ac{TSO} can use \eqref{P90:TSO} to adjust the P90 violation frequency, $\epsilon$, and conservativeness, $\theta$, to maximize its flexibility procurement.

First, we briefly introduce the simulation setup, and explain how \ac{EV} consumption is modelled. We then present the main findings of \textit{(i)} and \textit{(ii)}.

\subsection{Simulation setup}\label{sec:sim-setup}

As mentioned, Definition \ref{def:P90} and Problem \eqref{P90:General:DRJCC-tract} is agnostic towards any stochastic flexible resource with respect to bidding in Nordic ancillary services. For this case study, we consider an aggregator of \acp{EV} with flexibility to turn their power consumption up or down, and $p_{m}^{\text{B}}(\xi)$ represents the uncertain \ac{EV} portfolio baseline power consumption.

Figure \ref{fig:drjcc_raw} illustrates the simulation results for a portfolio comprised of 200 \acp{EV}. Weekend days are simulated to have higher consumptions on average, and the same is the case for evenings. Intentionally introducing non-stationarity, the average charge time grows proportional to the square root of time, consequently introducing a positive drift the average portfolio consumption. This scenario could symbolize, for instance, the effect of colder weather, where \acp{EV} require longer charging periods due to increased energy demand.

\begin{figure}[t]
    \centering
    \includegraphics[width=\columnwidth]{../figures/drjcc_raw.png}
    \caption{Simulation of \ac{EV} power consumption for 90 days. First day is a burn-in period, green is the \ac{IS} period (corresponding to the empirical distribution), and yellow is the \ac{OOS} period.}
    \label{fig:drjcc_raw}
\end{figure}

The \ac{IS} period in Figure \ref{fig:drjcc_raw} thus defines the empirical distribution under which Problems \eqref{P90:General:DRJCC-tract} and \eqref{P90:TSO} are solved, and the bids are evaluated on the \ac{OOS} period. This is a stylized way of intentionally highlighting the efficacy of distributional robustness for \ac{EV} consumption, but it generalizes to any situation where any flexible provider might have mis-specified its empirical (or predictive) distribution for any technology with stochastic baseline power.

\subsection{Aggregator bidding}

Figure \ref{fig:drjcc_bids} shows bids from an aggregator of 20 \acp{EV} and with $|\mathcal{I}| = 30$ with respect to their \ac{IS} available flexibility (as shown in blue) for different values of conservativeness $\theta$. Starting with the \ac{IS} period (left), all three bids are adhering to Definition \ref{def:P90}. For $\theta = 0.01$, the bids almost represent the 90\% quantile, i.e., the maximum allowed violation frequency. However, increasing $\theta$ decreases the bids (and therefore aggregator profits). But, as seen in the \ac{OOS} period (right), the bids for $\theta = 0.35$ are the only ones that (barely) adhere to the P90 rule. This is obviously due to the non-stationarity introduces as seen in Figure \ref{fig:drjcc_raw}.

Clearly, a low $\theta$ is risky in case the empirical distribution of available flexibility is just a little mis-specified. This might very well be the case other technologies as well, e.g., for volatile wind and solar power production or other flexible demand prone to sudden disturbances. This highlights the awareness aggregators should have when bidding, but also how the Definition \ref{def:P90} can be exploited to artificially allow for too high bids. This essentially corresponds to \textit{overfitting} to the empirical distribution as in the machine learning literature \cite{bishop2006pattern}.

\begin{figure}[!t]
    \centering
    \begin{adjustbox}{width=\columnwidth}
        \input{../figures/drjcc_bids_paper.tikz}
    \end{adjustbox}
    \caption{\textbf{Left}: Bids created on \ac{IS} empirical distribution of available flexibility. \textbf{Right}: Bids evaluated on \ac{OOS} distribution for Problem \eqref{P90:General:DRJCC-tract} with $\theta = \{0.01, 0.1, 0.35\}$. Bids are created for a subsample of 20 \acp{EV} from the portfolio described in \ref{sec:sim-setup}. In blue is shown the \ac{EV} portfolio available flexibility and its 10-90\% quantiles.}
    \label{fig:drjcc_bids}
\end{figure}



\subsection{\ac{TSO} flexibility procurement}\label{sec:motivation}

Figure \ref{fig:tso} shows the \ac{TSO} procurement for different values of $\epsilon$ and $\theta$ when solving \eqref{P90:TSO}. First of all, it is clearly seen how a high $\epsilon$, (i.e., a high violation frequency) only yields more available flexibility for the \ac{TSO} if $\theta$ is also high. This is due to $\nu$ in \eqref{P90:TSO} which penalizes lack of capacity. Interestingly, one can also clearly see a path of optimal $(\epsilon, \theta)$ starting from $(\epsilon, \theta) = (0.01, 0.01)$ to $(\epsilon, \theta) = (0.28, 0.05)$. This demonstrates that bid violations for high $\epsilon$ can be compensated by more conservative decision making, and still maintain a high flexibility procurement for the \ac{TSO}.

\begin{figure}[!t]
    \centering
    \includegraphics[width=0.95\columnwidth]{../figures/heatmap.png}
    \caption{\ac{TSO} flexibility procurement for given $\epsilon$ and $\theta$. Each $(\epsilon, \theta)$ has been solved using \eqref{P90:TSO} in a grid search on \ac{IS} data (see \ref{sec:sim-setup}). The optimal value is $(\epsilon, \theta) = (0.28, 0.5)$. }
    \label{fig:tso}
\end{figure}


\section{Conclusion}

This paper illustrated how stochastic flexible resources can participate in Nordic ancillary service markets using a \ac{DRJCCP} for bidding based on historical data of available flexibility. We showed how a tract formulation of the \ac{DRJCC} formulation allows for more robust bidding in non-stationary environments of the available flexibility. Furthermore, the perspective of the \ac{TSO} was also investigated with respect to their capacity procurement from such flexible resources. It was shown how an increased allowed violation frequency should be offset by a corresponding increase in conservativeness of bidding using the measure of the Wasserstein distance in the \ac{DRJCC}. These findings were exemplified using a simulated case study of an aggregator of \acp{EV} bidding their flexibility into an ancillary service with negligible energy.

For future work, it is of great interest to investigate the impact of a heterogenous portfolio of stochastic flexible resources with respect to \ac{TSO} procurement. As such, one could expect that flexible resources of different technology also respond differently when the \ac{TSO} prescribe violation frequencies and conservativeness. Moreover, our \ac{TSO} procurement in this work ignored prices, both for bids but also penalty prices for lack of capacity that was promised. Both are important from a system and societal perspective and should be included in future work. Lastly, an increase in supply of flexible resources lowers prices as well which will be offset by more uncertain supply. This trade-off is very interesting from a \ac{TSO} perspective, and should be studies further.


% \section*{Acknowledgement}

% The authors would like to thank...

\bibliographystyle{IEEEtran}

% \bibliography{tex/bibliography/Bibliography}
\bibliography{../bibliography/Bibliography}


\vfill

\end{document}
